\documentclass{article}
\usepackage[utf8]{inputenc}

\title{Sociale geografie}
\author{Vanden Driessche Théo }
\date{November 2017}

\usepackage{multicol}
\usepackage{eurosym}


\usepackage{geometry}
 \geometry{
 a4paper,
 total={170mm,257mm},
 left=20mm,
 top=20mm,
 }

\begin{document}

\maketitle

\section{De europese unie (vraagstukken in Europa}
\subsection{De geschiedenis van de Europese unie}

\subsubsection{Na tweede WO}
Na de tweede wereldoorlog besloten verschillende Europese leiders om samen te werken. Waarom kwamen ze tot dit besluit?\\
Na de negatieve ervaringen van wereld oorlog I en II, kozen verschillende Europese leiders om samen te werken.
\subsubsection{Geschiedennis van Europa}
\begin{itemize}
    \item 1949:\\
    Samen werking: Benelux. De oprichting van de BeNeLux beslissen economische samen te wwerken.
    \item 1951 De oprichting van de EGKS (Europese Gemenschap voor de Kolen en Staal):\\
    De landen van de Benelux + Frankrijk + Italië + West Duitsland.
    \item 1957: De verdragen van Rome:\\
    Euratom: Europese samen werking op vlak van atomenergie.\\
    EEG: Europese Economische Gemeenschap.
    \item 1967 EG:
    Europese Gemeenschap.
    \item 1973: 3 niewe landen voegen zich bij de EG:
    \begin{itemize}
        \item Denemark
        \item Ireland
        \item Groot-Britannië
    \end{itemize}
    \item 1981 nieuwe lidstaat: Griekenland
    \item 1986: Spanje en Portugal worden lid van de EG
    \item 1992: Het verdrag van Europese unie:\\
    In Maastricht (Nl) oprichting van de 3 pijlers:
    \begin{itemize}
        \item Vijligheid
        \item Politiek
        \item Defensie
    \end{itemize}
    \item 1995 : OOStenrijk, Finland, Zweden worden lid van EG
    \item 2002: invoering van de \euro als munt.
    \item 2004: 10 landen worden lid van de Europese Unie:
    \begin{multicols}{2}
        \begin{itemize}
            \item Cyprus: Nocosie
            \item Estland: Talinn
            \item Hongarije: Budapest
            \item letland: Riga
            \item Malta: Valetta
            \item Poland: Warschauw
            \item Tsequië: Praag
            \item Slovenië: Ljubljana
            \item Slovakijë: Bratislave.
        \end{itemize}
    \end{multicols}
    \item 2007: Bulgarije (Sofia) en Roemenie (Boekarest) worden lid van de Europese unie.
    \item 1 juli 2013: Kroatië (Tirana) wordt lid van de Europese unie.
    \subsubsection{Toekomst van de lieden van Europa}
    \item Welke landen zitten nog in de wachtkamer om lid te worden van de Europese Unie?
    \begin{multicols}{2}
        \begin{itemize}
            \item Turkije
            \item Macedonië
            \item Albanië
            \item Bosnië Herzegovina
            \item Kosovo
            \item Montenegro
            \item Servië
        \end{itemize}
    \end{multicols}
    \item Waarom mogen die landen niet onmiddelijk lid worden?\\
    De Kopenhagen criteria:
    \begin{itemize}
        \item Democratische regering hebben
        \item Goede mensenrecht hebben
        \item de minderheden moeten beschermd worden
        \item de economië moet goed fonctionneren
        \item de regels van de EU aanvarden
    \end{itemize}
\end{itemize}

    \section{de Europese instellingen}
    \subsection{Het Europese parlement}
    \subsubsection{Waar vergaderen ze?}
    \begin{itemize}
        \item Leopoldruimte in Brussel België
        \item Louise Weiss-gebouw straatsburg (Frankrijk) symbollische plaats.
        \item Secretariaat in Brussels en Luxemburg
    \end{itemize}
    \subsubsection{Wat zijn de taken van de Europese Parlement}
    \begin{enumerate}
        \item Wetgeving:
        \begin{itemize}
            \item Het Europees Parlement keurt de EU-wetgeving toe op basis van de voorstellen van de Europese comissie
            \item Het Europees Parlement beslist over internationale overeenkomsten
            \item Het Europees parlement beslist over toetreding van de nieuwe lidstaten
            \item Het Europees parlement bekijkt het werkprogramma van de Europese Comissie?
        \end{itemize}
        \item Toezicht:
        \begin{itemize}
            \item Het Europees Parlement houdt toezicht op alle EU-Instellingen.
            \item Het Europees Parlement kiest de voorzitter van de Europese Commissie?
            \item Het Europess Parlement behandelt petites van de burgers en kan een parlementaire enquête instellen.
            \item Het Europees Parlement voert overleg met de Europese Centrale Bank over het monettaire beleid van de Europese Unie.
            \item Het Europees Parlemetn stelt vragen aan de Europese Commissie en aan de Europese Raad.
            \item Het Europees Parlement stuurt waarnemings commissies naar verkiezingen.
        \end{itemize}
        \item De begroting:
        \begin{itemize}
            \item Het EP stelt de begroting van de Europees Unie op (samen met de Europee Raad)
            \item Het EP keurt de EU-begroting op lange termijn goed.
        \end{itemize}
    \end{enumerate}
    
        \subsubsection{De Verschillende fracties in het Europees parlement:}
        De leden van het parlemnt zijn gedeeld in fracties op basis van hun politieke kleur, niet op basis van hun nationaliteit.

    \subsubsection{Hoeveel leden?}
    751 leden, afhankelijk van de aantal bewonners van het land.
    
\subsection{De Europese Commissie}
\subsubsection{Waar vergaderen ze?}
Ze vergaderen in het Berlaymontgebouw in Brussel.
\subsubsection{Voorzitter?}
Jean-Claude Juncker
\begin{itemize}
    \item Benoeming: De staatshoofden en regeringsleiders in de Europese Raad stellen een kandidaat - voorzitter voor, op basis van de resultaten van de verkiezingen van het Europees Parlement. De kandidaat moet de steun van de meerderheid van het Europees Parlement krijgen om te kunnen worden benoemd.
\end{itemize}
\subsubsection{Taken van de Europese Commissie}
\begin{enumerate}
    \item De commissie stelt nieuwe wetgeving voor
    \item de Commissie beheerst het EU-beleid en wijst EU financering
    \item Handhaaft de EU-wetgeving: Ziet er samen met het Hof van Justitie op toe dat de EU-wetgeving in alle EU-lande juist wordt toegepast.
    \item Spreekbuis van de EU in de wereld.
\end{enumerate}
\subsubsection{Hoeveel leden? Wie zijn de leden?}
\begin{itemize}
    \item Een team of "College" van commissarissen één uit elk EU-Landen
    \item voorzitter: Juncker
    \item De voorzitter bepaalt wie verantwoordelijk is voor welk beleidsterrein.
\end{itemize}
    
\subsection{De Europese Raad = De Europese Top}
\begin{itemize}
    \item De vergaderingen van de Europese raad zijn topconferenties waarop de EU leiders beslissingen nemen over beleidsprioriteiten en belangrijke initiatieven.
    \item 1 vaste voorzitters sinds 2009:
    \begin{itemize}
        \item 2009-2014: HErman van rompuy
        \item 2014-2019: Donald Tusk
    \end{itemize}
    \item vergaderen normaal 4X per jaar in Brussel
    \item 28 leden -> 28 lidstaten
\end{itemize}

\subsection{De raad van Ministers}
\begin{itemize}
    \item 1 lid per lidstaat
    \item bvb: de bijeenkomst van alle Ministers van Landbouw.
    \item Om de 6 maanden is een ander land gastland 
\end{itemize}

\subsection{Het Europees hof van Justicie}
\begin{itemize}
    \item Regelt de conflicten tussen de verschillende lidstaten
    \item Ze zien erop toe dat de EUropese Wetgeving in elke lidstaat het zelfde wordt uitgevord
    \item Zetel in Luxemburg.
\end{itemize}

\section{Brexit}
\subsection{Definite}
Br-Exit:Britain-Exit Europa.
\subsection{Voorgeschiedenis}
\begin{itemize}
    \item 1973 kom GB in de EG.
    \item  Hun lidmaatschap was niet altijd evident
    \item Weigerden om lid van de eurozone worden
\end{itemize}
\subsection{Nigel Farage}
De UK Independance Party (UKIP) pleitte al langer voor een uitstap uit de Europese
Unie. Bij de Europese verkiezingen in 2014 kreeg deze partij ongeveer een kwart
de stemmen.\\
\par

De UKIP is een liberale (op economisch vlak) en eurosceptische politieke partij die
sinds 1993 bestaat. Ze hebben 22 fractieleden (van de 71) in het Europees Parlement\\
\par
Het bekendste lid van de partij is Nigel Farage.\\
\par
Nigel Farage (Farnborough, 3 april 1964). Hij is een Brits politicus en sinds 1999 is hij
lid van het Europees Parlement. Hij staat bekend om zijn controversiële uitspraken.
Hij beledigde ondermeer de Europese president Herman van Rompuy. Hij zei dat hij
het charisma had van een natte dweil en de uitstraling van een tweederangs
bankmedewerker. Hij noemde België een 'non-country'. Hij werd hiervoor 'beboet'.\\
\par
De voormalige premier van het VK, David Cameron, beloofde na zijn herverkiezing in 2015 om een bindend referendum over het lidmaatschap van de Europese Unie te
houden. Dit was in overeenstemming met een belofte uit het verkiezingsprogramma
van de Conservatieve Partij (de partij van Cameron). Hij dacht om op die manier de macht van de UKIP te verminderen. Dit was duidelijk een vergissing.

\subsection{David Cameron}
David Cameron (9 oktober 1966, Londen) is een Brits pollticus. Hij was van 11 mei
2010 tot en met 13 juli 2016 premier van het Verenigd Koninkrljk. Cameron studeerde
aan het bekende Eton College, een prestigieuze jongensschool. Hij studeerde
economie en politiek aan de Oxford Universiteit. Cameron behoort tot de conservatieve
partij maar hij wordt meer gezlen als een centrum figuur.\\
\par
Een ander bekend lid van de Conservatieve Partij is Boris Johnson (New York, 19 juni
1964). Hij is een Brits historicus, schrijver en politicus. Momenteel is hij Minister van
Buitenlandse Zaken in de regering May. Tussen 1 mei 2008 en 7 mei 2016 was hij
burgemeester van Londen. Nu is Sadiq Khan de burgemeester van Londen. Johnson
was een grote voorstander van de Brexit.\\

\subsection{Referendum}
De referendum werd op 23 juni 2016 door het Britse parlment georganiseerd. Een kleine meerderheid van 51.9\% koos voor uittreding.

\subsection{Gevolgen}
Voire feuille.

\section{Spanje: een land in crisis}
\begin{itemize}
    \item 2006: Begin van de crisis
    \item 2009-2010: Hoogste punt van de crisis
    \item 2011 15 mei: Beweging in Madrid, opstand tegen kapitalisme.\\
    Indignados, de jeugd oproep om te revolteren. Gebruik van geweld (maakt veel critic)
    \item Podemos: politic groep ("We kunnen") links geïnspireerd.
\end{itemize}
\paragraph{Stéphane Hessel}
\begin{itemize}
    \item Schreef een boek : "Indignez-vous" in oktober 2010
    \item vocht voor de mensenrechten, mensen waardigheid wordt gerespecteerd
    \item Zijn boek kwam een paar manden voor de eerste revolte in Europa: Spanje, Italië, Slovenië
\end{itemize}

\paragraph{Occupy Wallstreet}
\begin{itemize}
    \item Jongeren (Student) in Europa komen in opstand tegen re grote banken.
    \item Ze willen meer socialisatie
    \item Kreeg veel kritiek door het gebruik van geweld.
\end{itemize}

\end{document}
